\documentclass[runningheads]{llncs}
\usepackage{graphicx}
\usepackage{lipsum}
\usepackage{amsmath}

\begin{document}

\chapter{[Chapter 1] Introduction}



Physics Engine attempts to simulate real life physical properties through well-known physics laws.

They are commonly used in video games, since many games try to resemble what we already have in the real world.

In general, this is done with a physics engine, encapsulating most physics simulation modules.

They are then incorporated into the game engine itself, allowing scripts to easily control the movement and settings of important game objects.

The most popular game engine, Unity, has its own implementation of physics engine.

It then becomes significantly easier for game developers to simulate realistic effects. For example, to create "Flappy Bird" in unity, the bird could be attached with a rigid body physics component, which could then modify the movement of the bird according to gravity with response to player controls. The underlying physics engine sees the rigid body, and attempts to modify its position and rotation according to physics law. This engine integration has become an important tool of game development.

Motivation

Nowadays, many developers choose to rely on whatever comes as built-in for their choice of engine. Unity, for example, has an integration of the NVIDIA PhysX engine, which is provided under a freeware license. Other popular choices include many open-source physics engines, xxx. These ready-made tools are well-polished by big communities and have in-depth support for most functionalities the developers want. However, it's also not rare to see people create their own custom physics engine, as this allows for greater customization possibilities, and some ideas might eventually get contributed to the open-source community. I also take interest in setting up a physics engine from scratch of my own, additionally granting me learning experiences for underlying physics mechanics and software engineering problem solving.


While physics engine consists of many possible components that are considered on-going research problems, a main building block of it - Rigid body dynamics, is more widely used and accepted, with many well-established implementations and experiments. Rigid body dynamics studies systems of interconnected bodies which do not deform under applied forces. Relative position is preserved inside each rigid body, making it easy to calculate positions relative to the world. Despite the rigid assumptions imposed on the physical objects, rigid body simulation gives rise to quick use of well-known physical laws and common experiments, which would set the core fundamentals for this project.

\noindent\rule{12cm}{0.4pt}

\end{document}